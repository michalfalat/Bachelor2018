\chapter{Firma} % chapter* je necislovana kapitola
%\addcontentsline{toc}{chapter}{Úvod} % rucne pridanie do obsahu
\markboth{Firma}{Firma} % vyriesenie hlaviciek

\section{Odborné zameranie firmy}
Firma M2M solutions s.r.o \cite{firma} pôsobí na slovenskom  IT  trhu približne od roku 2010. Veľkosťou sa radí medzi malé až stredné  firmy s počtom zamestnancov 25-35 ľudí. Hlavným zameranim je vývoj IT riešení pre priemysel  a logistiku, vrátane vývoja  softwaru aj hardwaru. Firma sa tiež venuje aj úprave firemných procesov  na zvýšenie efektivity práce.  Medzi zákazníkov patria aj~mnohé nadnárodné firmy.  Kvôli rozšíreniu  pôsobenia  firmy  sa  v roku 2017 otvorila  aj nemecká pobočka pod názvom Werks Revolution. Do budúcnosti firma plánuje otvoriť ďalšie pobočky na~Slovensku a v Poľsku.
Firma má  vo svojom portfóliu viacero produktov. Medzi nosné produkty patria  systémy riadenia skladu,  manipulácia s tovarom podľa svetla, transportné systémy a~systém na  plánovanie výroby. Každý produkt je upravený na mieru podľa potrieb zákazníka. Firma sa snaží sledovať súčasné trendy a v budúcnosti  pripravuje väčšinu svojich produktov  vyvíjať univerzálne a presunúť ich do cloudu.  V súčasnosti sa firma zameriava aj na Industry 4.0 a vývoj a využitie IoT zariadení v priemysle. 

\section{Pracovné zaradenie}
Moje pracovné zaradenie  spočívalo hlavne vo vývoji a testovaní rôzných softvérových komponentov. Nakoľko som nemal  dostatok praxe, väčšinou som pracoval spolu s ďalšími programátormi v tíme, ktorí mi pomohli v prípade nejasnosti  a naviedli ma na správnu cestu. Z veľkej miery  sa jednalo o vývoj v ASP.NET C\#. Približne v polovici praxe som začal pracovať ako javascript developer vo frameworku Angular, s ktorým som pracoval približne  4 mesiace. Ďalšiu časť praxe som vyvíjal mobilnú aplikáciu  pre android na komunikáciu s bezdrôtovými tlačidlami.  Na konci praxe som bol zapojený do akvizície na návrh a vývoj systému pre navigáciu a lokalizáciu objektov v budovách.  Na tomto projekte bolo mojou úlohou nájsť všetky dostupné riešenia, ktoré by  sa dali použiť. V prvej fáze bolo tiež mojou úlohou zozbierať dáta zo senzorov v mobile a~implementovať Kalmanov filter na minimalizáciu šumu z meraných hodnôt.