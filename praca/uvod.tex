\chapter{Úvod} % chapter* je necislovana kapitola
%\addcontentsline{toc}{chapter}{Úvod} % rucne pridanie do obsahu
\markboth{Úvod}{Úvod} % vyriesenie hlaviciek

Moja odborná prax prebiehala vo firme M2M solutions s.r.o. Počas praxe som mal možnosť vyskúšať si prácu v tíme na viacerých zaujimavých projektoch v priemyselnom odvetví. V týchto projektoch som dostal možnosť naučiť sa veľa nových technológii, ale zároveň som chcel zúžitkovať aj moje vedomosti získané v počas školského štúdia. Hlavným dôvodom pre výber praxe bolo ziskanie preukázatelných praktických znalostí. Keďže viem, aká je prax pre informatika dôležitá a takmer každému zamestnávatelovi nestačia iba teoretické vedomosti získané zo školy. Aj pri~nástupe na prax ako brigádnik, som už musel vedieť preukázať nejaké jednoduché projekty, na ktorých som pracoval napríklad vrámci školských projektov aspoň pre základné overenie mojich praktických znalostí z programovania.  Odbornú prax vo firme M2M solutions s.r.o som si vybral  tiež  kvôli tomu, že som v tejto firme začal pracovať ešte pár mesiacov pred začatím samotnej odbornej praxe a chcel som v nej pokračovať aj naďalej, keďže mi vyhovovala aj z hľadiska dochádzania do práce. Počas praxe  som spoznal veľa skúsených programátorov, ktorí mi vedeli pomôcť s niektorými ťažšími  úlohami a veľa som sa od nich počas praxe naučil. Od praxe som očakával najmä naučenie sa nových  moderných technológii a trendov v programovaní,  získanie pracovných návykov ale aj získanie všeobecného prehľadu v IT odvetví, v ktorom  by som chcel  po skončení školy pracovať. Počas praxe som sa zároveň chcel zdokonaliť aj v anglickom jazyku, ktorý je pre informatika v dnešnej dobe  nevyhnutnosť, bez ktorej sa na pracovnom trhu nedá presadiť.


