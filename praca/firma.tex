\chapter{Firma} % chapter* je necislovana kapitola
%\addcontentsline{toc}{chapter}{Úvod} % rucne pridanie do obsahu
\markboth{Firma}{Firma} % vyriesenie hlaviciek

\section{Popis firmy}
Firma M2M solutions s.r.o \cite{firma} pôsobí na slovenskom  IT  trhu približne od roku 2010. Veľkosťou sa radí medzi malé až stredné  firmy s počtom zamestnancov 25-35 ľudí. Hlavným zameranim je vývoj IT riešení pre priemysel  a logistiku, vrátane vývoja  softwaru aj hardwaru. Firma sa tiež venuje aj úprave firemných procesov  na zvýšenie efektivity práce.  Medzi zákazníkov patria aj nadnárodné firmy napriklad DHL , Whirlpool, Gefco, Pastorkalt a iní.  Kvôli rozšíreniu  pôsobenia  firmy  sa  v roku 2017 otvorila  aj nemecká pobočka pod názvom Werks Revolution. Do budúcnosti firma plánuje otvoriť ďalšie pobočky na Slovensku a v Poľsku.

\section{Produkty firmy}
Firma má  vo svojom portfóliu viacero produktov. Medzi nosné produkty patria WMS  systém riadenia skladu ,  M2L - manipulácia s tovarom podľa svetla , transportné systémy a systém na  plánovanie výroby. Každý produkt je upravený na mieru podľa potrieb zákazníka. Firma sa snaží sledovať súčasné trendy a v budúcnosti  pripravuje väčšinu svojich produktov  vyvíjať univerzálne a presunúť ich do Cloudu.  V súčasnosti sa firma zameriava aj na Industry 4.0 a vývoj a vuužitie IoT zariadení v priemysle. 

\section{Pracovné zaradenie}
Moje pracovné zaradenie  spočívalo vo vývoji a testovaní rôzných softvérových komponentov. Nakoľko som nemal  dostatok praxe, väčšinou som pracoval spolu  s ďalšími programátormi v tíme. Spočiatku sa jednalo o jednoduché implementačné veci v ASP.NET C\#  no postupom času  som začal pracovať ako FrontEnd Developer vo frameworku Angular4 , s ktorým som pracoval približne polovicu odbornej praxe. Ďalšiu časť praxe som vyvíjal mobilnú aplikáciu  pre android na využitie IoT v priemysle.  Je teda zrejmé , že som pracoval s viacerými technológiami, čo mi zároveň dáva aj všeobecný prehľad v IT oblasti.
